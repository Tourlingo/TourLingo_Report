\chapter*{\vspace{-3cm}Abstract}
\addcontentsline{toc}{chapter}{Abstract}
\vspace{-0.5cm}   % pull text up

This report presents the design, development, and evaluation of \textit{Tourlingo}, a Multilingual Virtual Tour Assistant web application created in collaboration with Al Fateh Technologies. It is basically a web application created to solve common problems experienced by international tourists especially who are not much fluent with English or the native language of the place. The web app incorporates few key features like map customised to cater needs of tourists, information about the local culture and more importantly translation of the contents provided to on the web pages. The web app is primarily designed considering needs of tourists using smartphones during their travel. However, it has responsive interface which works well on other devices like desktop and tablets.

%%%% This paragraph is not that impressive %%%%
The map page on the website shows points of interests (POIs) which are categorised in 2 side panels - essentials and recommendations. In those panels, further subcategories are used to show nearby places. These places are showed on the map with the help of custom pin icons for better visualisation. Moreover the map page can give directions to the searched place in user's own selected language. Another important feature - AI-powered cultural tips are fetched for user's current city or selected city. Moreover, the AI generated text is translated in real-time by LibreTranslate engine at backend of the web app.

Overall, we not only planned frontend and backend architecture but also evaluated and redesigned multiple times thorughout the period of 3 months. A microservices-based backend, implemented with Spring Boot, was iteratively refined across four architectural versions, achieving a 20x increase in throughput and a 99.45\% reduction in average response time through database caching and horizontal scaling. On the frontend, Next.js, TypeScript, TailwindCSS, and Zustand were employed to deliver a scalable and maintainable user interface, with modular overlays for essentials and recommendations. Integration with Overpass API, Nominatim, and OSM tile servers ensured robust geospatial functionality.

Furthermore, we not only focused on just the product defined by the customer using a Product Requirements Document (PRD), we followed agile practices which we learnt throughout our two terms at the university and especially in Software Engineering module. Starting from creating digital prototype to validating interface using think-aloud evaluation, we documented all the steps and tried data-driven decision making. The important feedback and validation we got was from the 2 testathons organised by the department of computer science. We had compiled the structured usability feedback of more than 30 testers and focused on the main raised issues or required enhancements after each of the testathons. Our product gained actually benefitted by this kind of user feedback which was evident cause of the increase in rating achieved for features and interface in second testathon compared to the first one.

 Consequently, Tourlingo delivers value on two primary fronts. First, it serves as a practical tool designed to make travel more accessible and inclusive by breaking down language and cultural barriers for tourists. Second, it demonstrated how effective software can be built through a user-first design philosophy, agile teamwork, and efficient full-stack system architecture. The roadmap for future development includes the expansion of language coverage, enrichment of cultural and local discovery content, the integration of premium APIs to improve quality and speed of data fetched.