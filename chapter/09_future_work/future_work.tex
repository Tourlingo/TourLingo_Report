\chapter{Future Work}
\section{Technical Improvements}
Our MVP implementation successfully delivers the core features outlined in the initial product vision. However, during development, we encountered several technical constraints that limited functionality and user experience, we plan to improve in the future. These include:
    \begin{itemize}
        \item \textbf{API Limitations}: As we are currently using free-plan APIs for translation, geolocation, and cultural data, we faced limitations in both request frequency and data richness. We hope to explore more advanced or commercial APIs to enhance responsiveness and improve the quality of information delivered to users.
        \item \textbf{Dynamic Navigation:} At present, our application provides static route navigation. In future development, we aim to implement real-time, dynamic navigation features that adapt to the user’s current location and preferred mode of travel.
        \item \textbf{Richer Data for Locations:} Currently, many places display default or limited content due to lack of data. We plan to enrich the user experience by integrating more comprehensive data, such as detailed descriptions and accurate images.
    \end{itemize}
\section{Feature Expansion Ideas}
Through feedback collected during 2 Test-a-thons \textbf{\textcolor{red}{(add reference of the Testathon report)}} and our own expectation of the product, we identified several promising directions that could further enhance the application in future development stages:

\subsection{Safety and Security Features}
We plan to explore the integration of features that help users identify potentially unsafe areas by providing contextual awareness and travel advisories. In addition, we aim to enhance the visibility and accessibility of emergency services by visually highlighting key locations such as police stations, hospitals, and embassies directly on the map. These improvements are intended to support users in quickly accessing assistance during urgent situations, especially in unfamiliar environments.

\subsection{Cultural and Local Discovery Enhancements}
To deepen users’ cultural immersion and local engagement, we hope to expand the recommendation system to promote independent local businesses and lesser-known points of interest, often referred to as ‘hidden gems’. In future updates, the cultural tips section may include more region-specific etiquette, visual guides, local slang, and real-time information about ongoing events. These additions are intended to help travellers better adapt to the cultural environment and engage with their surroundings more meaningfully.

\subsection{AI and Search Improvements}
Another potential enhancement involves the integration of an AI assistant capable of responding to natural, conversational queries. This would allow users to ask questions such as ‘Where can I find a vegetarian restaurant nearby?’ and receive tailored results. To further improve the user experience, we also aim to implement a more corrective search function with typo-tolerance and contextual awareness. This would ensure that users receive accurate and relevant suggestions even when inputting partial or slightly incorrect queries.

\subsection{Trip Planning Tools}
We hope to incorporate more intelligent trip planning functionalities into future versions of the application. In particular, we aim to enable the map to automatically generate efficient multi-stop travel routes, reducing unnecessary detours and optimising the sequence of visits. Additionally, we envision allowing users to input their intended trip duration (e.g., number of days), upon which the system could suggest personalised itineraries including destinations, estimated travel times, and cultural activities. Furthermore, we are considering features that allow users to leave comments or ratings on places they visit, fostering a community-driven travel experience and encouraging knowledge-sharing among international tourists.

