\subsection{Conclusion}

The iterative development of the Tourlingo backend architecture successfully achieved the primary objectives of cost efficiency, scalability, and performance optimisation. The systematic approach, guided by empirical performance analysis and constraint-driven decision making, resulted in a robust system capable of supporting the application's multilingual requirements.

\subsubsection{Key Contributions:}
\begin{itemize}
    \item Demonstration of effective cost-optimization strategies in translation service architecture.
    \item Validation of iterative development methodology for complex system optimization.
    \item Implementation of scalable microservices architecture supporting linear performance growth.
    \item Integration of caching strategies achieving significant operational cost reductions.
\end{itemize}

\subsubsection{Technical Achievements:}
\begin{itemize}
    \item 99.45\% response time improvement through cache implementation.
    \item Cost reduction by improving the business logic from per user AI agent and translation call to per city AI agent call and strategic caching approaches.
\end{itemize}

% \subsubsection{Professional Development Insights:}
% The project demonstrated the critical importance of constraint-driven architecture design, where budgetary limitations led to innovative technical solutions that ultimately provided both cost and performance benefits. The iterative methodology proved essential for managing system complexity while maintaining development momentum and system reliability.

% The architecture provides a solid foundation for future enhancement while maintaining the core principles of cost efficiency and scalability that guided the initial design decisions. The systematic documentation and performance analysis establish a framework for continued evolution and optimization as the system requirements grow and change.