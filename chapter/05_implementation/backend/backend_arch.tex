\section{Design and Implementation of Backend}

\subsection{Introduction}
This chapter provides details about the design and implementation of the back-end architecture for Tourlingo. The backend's primary responsibilty is to deliver near real-time translations of text into the user's preferred language. The backend was developed to serve the minimum requirement of providing a translated text to the frontend in 
%%% Need to check if i can use json in this way
JSON format using a translation engine and progressively enhanced to satisfy both functional and non-functional requirements. This chapter highlights the design decisions taken to implement a scalable, cost-effective back-end to meet system requirements.

\subsection{Chapter Objectives}

This chapter aims to:

\begin{itemize}
    \item Analyze functional and non-functional requirements that influenced architectural decisions
    \item Justify technology selection through systematic evaluation frameworks
    \item Document the iterative development methodology and its effectiveness
    \item Evaluate performance improvements achieved through architectural evolution
    \item Assess trade-offs inherent in each design iteration and their long-term implications
\end{itemize}

\import{chapter/05_implementation/backend/B_Requirements}{b_requirements.tex}

\import{chapter/05_implementation/backend/B_technology_selection}{b_tech_selection.tex}

\import{chapter/05_implementation/backend/B_architectural_design}{b_arch_design.tex}

\import{chapter/05_implementation/backend/B_architectural_design}{b_iterative_arch_evolution.tex}

\import{chapter/05_implementation/backend/B_architectural_design}{b_evaluation.tex}

\import{chapter/05_implementation/backend/B_architectural_design}{b_future_enhancements.tex}

\import{chapter/05_implementation/backend/B_architectural_design}{b_conclusion.tex}