\chapter{Introduction}

\section{Motivation}

While tourism is one of the world's fastest-growing industries, international travel isn't always straightforward. Visitors to new destinations often run into common hurdles like language barriers, unfamiliar local customs, and a lack of digital tools that really meet their needs. Many popular navigation applications are built with English speakers in mind, which creates accessibility problems for a global audience. At the same time, most translation tools are stand-alone apps that don't integrate smoothly with navigation systems. Even major travel websites, which offer some translation, often assume users already know about local customs and etiquette. This project was initialised to fill that gap.

We set out to develop a web app that acts as a multilingual virtual assistant for tourists. Our app combines map-based navigation, real-time translation, and culturally relevant guidance into a single, integrated tool. To build it, we used OpenStreetMap for map data \cite{osm}, LibreTranslate as a translation engine \cite{libretranslate}, and a custom AI integration to provide cultural tips. Our ultimate goal is to help tourists feel more confident and prepared as they explore, making international travel a more inclusive and enjoyable experience for everyone \cite{inclusiveTourism}.

\section{Project Aims and Objectives}

The main goal of this project was to design and build a web application that works across different devices and serves as a travel assistant. We planned to achieve this by integrating three core features: map navigation, automated translation, and cultural guidance.

\textbf{Product Objectives}
\begin{itemize}
\item To ensure the app has a responsive design that works reliably on mobile, tablet, and desktop devices.
\item To build an interactive map that highlights points of interest (POIs) for tourists, with filters for categories like restaurant, store and pharmacy.
\item To add a translation feature for the user interface, ensuring that web page elements like side panels and directions are displayed in the user’s preferred language.
\item To offer helpful cultural tips (such as etiquette, key phrases, transportation and payment methods) using dynamically generated and translated content \cite{gpt_culturetips}.
\item To create a modular architecture where key components (like the map, POI data cards , or translation engine) could be replaced or extended later without a complete system restructuring.
\end{itemize}

\textbf{Team \& Learning Objectives}
\begin{itemize}
\item \textbf{Teamwork \& Collaboration:} We aimed to work effectively as a team by defining clear roles (e.g., project manager, dev lead), using shared task boards, and holding regular meetings to coordinate our work.
\item \textbf{Client Engagement:} We had weekly scheduled meetings with our industrial partner, Al Fateh Technologies, to have regular demos and feedback sessions making sure our final product met real-world industry needs.
\item \textbf{Software Engineering Practice:} We applied standard software engineering methods, including defining a minimum viable product (MVP), writing user stories, prototyping, and using an iterative development process.
\item \textbf{Technical Growth:} A key goal was to expand our technical skills by learning and using new frameworks, external services, and techniques for building a full-stack web application.
\item \textbf{Quality \& Testing:} We performed manual tests covering user acceptance testing, unit tests to maintain code quality and performance testing to test the application on normal load conditions.
\item \textbf{Professional Skills:} We committed to maintaining professional conduct by scheduling regular meetings with our client and academic supervisor, complete with clear agendas and minutes.
\end{itemize}

\section{Client Brief and Challenges}

Our client, Al Fateh Technologies, creates digital tools for the travel sector. The client's requirement was to develop an application to help tourists explore cities without needing to know the local language. The client wanted to have a website specifically for mobile, which can serve purpose of not only providing directions and nearby places in their own language, but also provide information about the city or town they are looking for. The main idea was to reduce the "culture shock" some travellers feel and improve their overall experience, helping them feel more comfortable and welcome \cite{culturegap}.

The initial brief was quite open, which gave us a lot of creative freedom but also meant we had to work hard to define and limit the project's scope. After a few meetings with the client, we narrowed our focus to three key features: intelligent map navigation, real-time translation, and AI-generated cultural advice.

We faced a few key challenges during the project:
\begin{itemize}
\item Getting the translation service to work fast enough for a smooth, real-time user experience.
\item Designing a map interface that was user-friendly and looked good, especially on mobile devices, while showing relevant, tourist-focused POIs.
\item Making sure the AI-generated cultural tips were accurate, relevant, and culturally sensitive, while also trying to mitigate potential biases.
\item Effectively coordinating a team of eight people within a tight project timeline, which we managed by using agile development methods.
\end{itemize}

One significant limitation was the lack of certain data, like images for points of interest, because we were relying on free, open-source APIs. To keep the project moving forward, we used placeholder images for MVP. If the client decides to invest more in the future, the app could be improved by integrating premium APIs to provide a richer user experience with up-to-date information \cite{crowdsourcing}.

\section{Scope and Deliverables}
This project's focus was to deliver a working prototype of the web application, responsive for both mobile and desktop. The scope included:
\begin{itemize}
\item A map interface built with OpenStreetMap API, featuring POIs relevant to tourists and filters for different categories.
\item Dynamic translation of UI static text, directions, and POI descriptions, using the LibreTranslate API.
\item A feature for AI-generated cultural tips tailored to the user’s selected location.
\item A cloud-hosted backend to handle data management and API integrations reliably.
\item Thorough testing and validation, including two "test-a-thons" where we gathered direct user feedback.
\end{itemize}

Due to time and resource constraints, we decided that a few features were out of scope: user-generated ratings and recommendations, translation of text on the map, and a system for automatically fetching images for all POIs. We have noted these as potential improvements for future development.

The final deliverables for this project are:
\begin{itemize}
    \item A functional web application, which we named "Tourlingo", successfully deployed for demonstration purposes and making it temporarily accessible through a public URL.
    \item The complete source code and all technical documentation, available in a public GitHub repository.
    \item A report detailing our test results and the user feedback we collected.
    \item A demonstration video that showcases the application's key features.
\end{itemize}