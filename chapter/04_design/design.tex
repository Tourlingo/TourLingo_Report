\chapter{Design}

\section{Introduction}

The design of Tourlingo aimed to design not just a map-based website, but a multilingual and cultural sensitive travel assistant. Our design process focused on helping international travellers feel more comfortable and confident in unfamiliar environments by providing accessible, local information in their preferred language.

To achieve our goal, we followed a user-centred design approach, using wireframing and interactive prototyping to explore key features. We then conducted early user evaluation through a think-aloud method, which informed several improvements to the initial prototype. The following sections describe our design requirements, UX process and early-stage evaluation in more detail.

\section{Design Requirements}

\subsection{Overview}
To ensure that our product effectively supports international tourists navigating unfamiliar environments, we began the design by analysing client's requirements and aligning them with user-centred design (UCD) principles. The client's initial product vision highlighted the importance of multilingual support, local cultural guidance and user-friendly user interface (UI). Based on these inputs, we carried out a systematic prioritisation process to define functional, technical and accessibility requirements.

\subsection{Client Side Requirements}
The project began with a clear set of high-level requirements proposed by the client, aiming to assist international tourists in navigating unfamiliar cultural and physical environment. These initial expectations provided the foundation for our design process. The key requirements are summarised as follows:

\begin{itemize}
    \item 
\end{itemize}

\subsection{User-Centred Design Requirements}

\subsection{Requirements Documentation and MVP}

\section{Wireframing and UX}
things to include in this section:

1. Purpose of the wireframing

2. tools and processes used

3. UX Considerations
\section{Prototyping and Early Evaluation}
things to include in this section:

1. Prototype approach: Figma, components design website

2. First user evaluation and collect insight

3. Finally, decide the first version of the prototype,e then start to implement

4. For this evaluation, focus on the design stage, distinguish from Chapter 6. Focus on early testing of wireframes and prototypes. What changes were made based on user input before implementation.