\chapter{Background and Related Work}
\begin{sloppypar}

\section{Problem Context}
Tourists arrive in a new city and even simple things choosing the right route, understanding station signage and opening hours, or asking for directions often turn into a scavenger hunt across apps. One app shows the map, another helps with basic phrases, blogs explain local manners, and separate searches find essentials like ATMs or hospitals. Using all these tools takes extra time and erodes traveller confidence \cite{mobileit2024}.

\noindent \textbf{Multilingual understanding.}
Even with “translate label” features in major map apps, place names, signs, timetables, opening hours, and descriptions often remain in the local language or appear in inconsistent translterations \cite{googlemaps,applemaps}. Our focus is on \emph{multilingual assistance} for the UI and in-app content panels (e.g., POI details), so travellers can understand what they are looking at without leaving the flow. For clarity, Tourlingo does not translate base map labels; the base map language follows the provider. Translations apply to the UI and in-app panels (e.g., POI details and cultural tips).

\medskip

\noindent \textbf{Cultural fit.}
Guidebooks and blogs provide broad etiquette, but rarely the city specific cues that matter on the ground \cite{lonelyplanet,culturetrip}. Concise, city-based cultural tips help visitors avoid Mistake and feel more confident engaging with locals.

\medskip

\noindent \textbf{Fewer apps, fewer detours.}
A typical journey still involves hopping between a map, a translator, and the web. Every switch breaks focus, drains battery, and prolongs simple tasks; on mobile, context switching alone can add roughly 20\% to task time \cite{mobileit2024}. Reducing that friction is pivotal for short trips where minutes matter.

\medskip

\noindent \textbf{Finding essentials when it matters.}
Accessing police stations, hospitals, ATMs, and toilets often means repeated, separate searches with uneven categories or freshness—frustrating at best and risky in urgent situations \cite{googlemaps}. A unified “travel essentials” view closes this gap so help is one tap away.

\medskip

These frictions create anxiety and missed opportunities: misunderstandings, breaches of local norms, and time lost to juggling tools instead of enjoying the place.

\section{Related Work and Existing Solutions}
\paragraph{Scope.}
The table below compares widely used navigation apps a traveller might choose instead of our system, summarising practical pros/cons and the gaps that matter for language/culture and “travel essentials.”

\begin{table}[H]
  \centering
  \caption{Representative Apps — Pros and Cons}
  \label{tab:per-app}
  \begin{tabular}{|p{3cm}|p{5.2cm}|p{5.2cm}|}
    \hline
    \textbf{App} & \textbf{Pros} & \textbf{Cons} \\
    \hline
    Google Maps \cite{googlemaps} &
      Comprehensive POIs, multi-modal routing (drive/walk/transit), reviews, offline areas. &
      Labels default to local language; “translate labels” coverage can be uneven; no in-flow cultural etiquette; essentials (police, hospitals, ATMs, toilets) require separate searches. \\
    \hline
    Apple Maps \cite{applemaps} &
      Smooth iOS integration, clear UI, privacy posture; improving transit coverage. &
      Fewer reviews/POIs in some regions; limited multilingual labelling consistency in some regions; no cultural tips layer. \\
    \hline
    HERE WeGo \cite{herewego} &
      Robust turn-by-turn and strong offline map packs; reliable driving focus. &
      Smaller review ecosystem; no built-in translation or cultural guidance; essentials via ad-hoc searches. \\
    \hline
    Citymapper \cite{citymapper} &
      Best-in-class public-transport UX in supported cities; live departures and disruption alerts. &
      Limited city coverage; no integrated translation; no cultural tips or unified essentials layer. \\
    \hline
    Maps.me \cite{mapsme} &
      Fully offline OSM-based navigation; lightweight; includes many basic POIs. &
      No integrated translation for UI/POI content; no cultural guidance; UI can feel dense; data freshness varies by region. \\
    \hline
  \end{tabular}

  \vspace{4pt}
  \footnotesize Availability and performance vary by country/city and app version; we focus on what matters for tourists facing language and culture gaps.
\end{table}

\noindent\textbf{Implication.}
These apps are excellent at routing and POIs, but they assume language familiarity, provide little in-flow cultural context, and lack a single, reliable “travel essentials” view—so travellers still end up context-switching. In this release, Tourlingo supports walking, cycling, and driving; public-transport routing is out of scope.

\paragraph{What is missing and what we add.}
Most mainstream navigation tools still lack integrated multilingual \emph{UI and content panels} (beyond system language), concise location-specific cultural cues, and a one-tap view of essentials such as police, hospitals, ATMs, and toilets. \emph{Tourlingo} combines navigation with multilingual UI and side-panel content, cultural tips, and an essentials layer in a single flow, reducing context switching for non-native speakers and first-time visitors.

\end{sloppypar}
